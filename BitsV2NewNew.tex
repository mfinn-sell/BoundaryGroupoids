We present a new approach to study the coarse Baum-Connes conjecture for spaces of graphs of bounded geometry, specifically providing new insight into counterexamples from \cite{higsonpreprint,MR1911663,explg1}. Recall that the coarse Baum-Connes conjecture states that a certain assembly map:
\begin{equation*}
\mu_{X,red}:KX_{*}(X) \longrightarrow K_{*}(C^{*}(X))
\end{equation*}
is an isomorphism for $X$ a uniformly discrete bounded geometry metric space. This conjecture is a geometric intrepretation of the well-known \textit{Baum-Connes conjecture} \cite{MR1292018}. For finitely generated groups the Cayley graph coming from some finite generating set will be a uniformly discrete space with bounded geometry, and a positive result for the coarse Baum-Connes conjecture in such situations has strong implications such as the Strong Novikov conjecture \cite{MR866507}, or the existence of metrics with positive scalar curvature.

The Baum-Connes conjecture can be developed in other directions, particularly into the realm of topological groupoids \cite{MR1798599}, which arise naturally in topology and noncommutative geometry to study objects such as foliations on manifolds or group actions on topological spaces. It is a well known result from \cite{MR1905840} that the above statement of the coarse Baum-Connes conjecture can be replaced with a conjecture with coefficients for some groupoid $G(X)$ that we can associate to any uniformly discrete bounded geometry metric space.

In this context, the coarse Baum-Connes conjecture states that the map:
\begin{equation*}
\mu_{r}:K_{*}^{top}(G(X), \ell^{\infty}(X,\mathcal{K})) \rightarrow K_{*}(\ell^{\infty}(X,\mathcal{K})\rtimes_{r}G(X))
\end{equation*}
is an isomorphism. In the paper \cite{MR1911663}, counterexamples were constructed for the coarse groupoid when the space was a \textit{expander graph} \cite{MR2569682}. This class of spaces has subsequently been well studied with respect to the coarse Baum-Connes conjecture \cite{higsonpreprint,MR2431253,MR2419930,MR2764895,MR2568691,explg1,explg2}. The key property associated to expander graphs is the ability to construct \textit{ghost projections} in the Roe algebra. These are infinite rank projections that cannot be approximated well by operators of very small propagation as they have essentially no local information. 

In the paper \cite{MR1911663} the method for constructing counterexamples to the conjecture for groupoids utilised a ``short exact sequence'' built from reductions of a groupoid -the main idea was that this sequence, whilst always exact for the maximal $C^{*}$-algebras of these groupoids may fail to be exact when we consider the left regular representation. The spotlight in \cite{MR1911663} was on the original groupoid, not the reductions, whilst in this paper we are going to consider specifically the groupoids that arise from reductions to \textit{boundaries} of the unit space. 

The object of this paper to give elementary proofs of many of these results concerning \textit{box spaces} of residually finite groups as well as proving a strengthened geometric version of the main results of Willett and Yu in \cite{explg1,explg2}. We obtain this framework by revisiting the ideas of \cite{MR1911663,MR1905840} and by introducing a boundary conjecture associated to a uniformly discrete bounded geometry metric space $X$:
\begin{conjecture}(Boundary Coarse Baum-Connes conjecture)\label{MC:S1}
Let $X$ be a uniformly discrete bounded geometry metric space and let $G(X)$ be the associated coarse groupoid on $X$. Then:
\begin{equation*}
\mu_{X,bdry}:K_{*}^{top}(G(X)|_{\partial\beta X}, l^{\infty}(X,\mathcal{K})/C_{0}(X,\mathcal{K})) \rightarrow K_{*}((l^{\infty}(X,\mathcal{K})/C_{0}(X,\mathcal{K}))\rtimes_{r}G(X)|_{\partial\beta X})
\end{equation*}
is an isomorphism.
\end{conjecture}

The intuition behind this conjecture is to quotient out by the ghost operators in the Roe algebra and then consider the K-theory of what remains. We formalise this idea in section \ref{Sect:CE}.

Now consider a space $X = \sqcup X_{i}$ constructed from a sequence of finite graphs $\lbrace X_{i}\rbrace$. The boundary coarse Baum-Connes conjecture for $X$ in this instance allows us, via the Five Lemma and \cite[Lemma 9]{MR1905840}, to conclude the coarse Novikov conjecture for $X$. We illustrate how this setting can be adapted to provide new proofs of results in \cite{explg1} concerning large girth expanders as well as the result of \cite[Theorem 5.7]{MR2764895} concerning box spaces of linear groups. The maximal conjecture is also considered thereby allowing us to get a new proof a Theorem of \cite{MR2431253,MR2419930,MR2568691}. This relationship makes it desirable to understand for which sequences this boundary conjecture is an isomorphism and to that end we prove:

\begin{theorem}\label{MT:S1}
Conjecture \ref{MC:S1} is true for the following classes of spaces:
\begin{enumerate}
\item Sequences of spaces that uniformly uniformly embed into Hilbert Space. 
\item Sequences of spaces that have large girth and uniformly bounded vertex degree.
\item Generalised Box Spaces associated to finitely generated groups with the Strong Baum-Connes Property.
\end{enumerate}
\end{theorem}
This theorem allows us to recover all known information about spaces of the form $X = \sqcup X_{i}$ for sequences of finite graphs $\lbrace X_{i} \rbrace$. Remarkably this includes some examples of groups with property (T), whose box spaces have geometric property (T) introduced by Willett and Yu in \cite{explg2}.

In the last section of this paper we provide a counterexample to the boundary coarse Baum-Connes conjecture defined in conjecture \ref{MC:S1}. We now recap all the major themes.


#################################################################################

The coarse Baum-Connes conjecture for metric spaces plays a central role in answering positively certain topological and group theoretic problems; it implies a positive solution to the Novikov conjecture for finitely generated discrete groups \ref{} and gives an answer to questions about positive scalar curvature metrics on manifolds \ref{}. It is well known that this conjecture has counterexamples \ref{} in the class of coarsely disconnected spaces.

In this paper we introduce a new conjecture associated to uniformly discrete bounded geometry metric spaces that is tailored to studying coarsely disconnected spaces and attempts to describe their asymptotic geometry. Phrased in terms of a certain Baum-Connes conjecture for a naturally constructed groupoid, it asks if:

is an isomorphism. We call this conjecture the boundary coarse Baum-Connes conjecture and we introduce it fully in Section \ref{}. The idea here is that this conjecture should be easier to prove in those instances that the coarse Baum-Connes conjecture fails.

The primary objective of the first half of this paper is to prove this conjecture for certain coarsely disconnected spaces:

\begin{thm}
The boundary coarse Baum-Connes conjecture holds for the following classes of coarsely disconnected spaces:
\begin{enumerate}
\item box spaces of finitely generated coarsely embeddable groups;
\item coarse disjoint unions constructed from sequences of finite graphs with large girth and uniformly bounded above regularity.
\end{enumerate}
\end{thm}

The first point is an easy observation outlined in Section \ref{}, however the second point is rather more interesting and relies on carefully studying the natural geometry of large girth sequences to produce a \texit{partial action} of free group. These ideas are covered in Section \ref{}.

The second objective of this paper is to connect the boundary coarse Baum-Connes conjecture, via homological methods, to the coarse Baum-Connes conjecture. Using this machinery we give elementary proofs of many of the counterexample arguments from \cite{} as well as many results concerning classes of expander graphs present in the literature \cite{}; in particular in Section \ref{} we give an elementary proof of results of Willett and Yu \cite{} concerning large girth sequences:

\begin{thm}
Let $X$ be a space of graphs constructed from a sequence of finite graphs with large girth. Then:
\begin{enumerate}
\item the coarse Baum-Connes assembly map is injective for $X$
\item If $X$ comes from an expanding sequence then the coarse Baum-Connes assembly map for $X$ is not surjective.
\end{enumerate}
\end{thm} 

Finally in Section \ref{} we give a counterexample to the boundary coarse Baum-Connes conjecture by considering a space introduced by Wang \cite{} and adaptions of the counterexample arguments present in the literature:

\begin{thm}
There is a space $Y$ constructed from a box space of $SL_{2}(\mathbb{Z})$ for which the boundary coarse Baum-Connes assembly map is injective but fails to be surjective.
\end{thm}