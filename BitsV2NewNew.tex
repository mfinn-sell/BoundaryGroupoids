\begin{lemma}\label{Lem:ParFree}
The partial action of $F_{k}$ defined above extends to $\beta X$ and is free on the boundary $\partial \beta X$.
\end{lemma}
\begin{proof}
Fix $g \in F_{k}$. As the sequence of graphs has large girth there exists an $i_{g} \in \mathbb{N}$ such that for all $i\geq i_{g}$ $\theta_{g}^{i}$ has no fixed points.  Let $\omega \in \widehat{D}_{\theta_{g}^{*}\theta_{g}}$ and assume for a contradiction that $\theta_{g}(\omega)=\omega$. Consider the graph $G$ with vertex set $V=\sqcup_{i\geq i_{g}} X_{i}$, where two vertices $x$ and $y$ are joined by an edge if and only if $y=\theta_{g}(x)$. This graph will have degree at most 2. Additionally, the set of non-isolated vertices inside $V$ contains $D_{\theta_{g}^{*}\theta_{g}}$ and so the subgraph with vertex set consisting of the non-isolated vertices and the same edge set is chosen by $\omega$ and has degree at most 2. Any such graph can be at most $3$ coloured, subsequently the vertex set breaks into three disjoint pieces $V_{1}, V_{2}$ and $V_{3}$ and the ultrafilter $\omega$ will pick precisely one of the $V_{i}$. The action of $\theta_{g}$ sends $V_{i}$ into $V_{i+1}$ modulo $3$ by construction. Lastly, $V_{i} \in \omega$ implies $V_{i+1} \in \theta_{g}(\omega)=\omega$, which is a contradiction.
\end{proof}

#########################################

\begin{proposition}\label{Prop:Ghost}
Let $\lbrace X_{i} \rbrace_{i\in \mathbb{N}}$ be a sequence of finite graphs. The the following hold for the space of graphs $X$:
\begin{enumerate}
\item The induced map $i_{*}:K_{*}(\mathcal{K}(\ell^{2}(X,\mathcal{K})) \rightarrow K_{*}(C^{*}X)$ is injective.
\item If $X$ is an expander then $K_{*}^{top}(X\times X)\cong K_{*}(\mathcal{K}(\ell^{2}(X)) \rightarrow K_{*}(I_{G})$ is not surjective.
\end{enumerate}
\end{proposition}
\begin{proof}
Consider the coarse map $X=\sqcup_{i} X_{i} \rightarrow P:=\sqcup_{i} \ast$ given by projecting each factor to a point and $P$ carries the coarse disjoint union metric. This induces a tracelike map on the K-theory of $C^{*}X$, which we denote by $Tr$. Observe also that $C^{*}P \cong \ell^{\infty}(P,\mathcal{K})+\mathcal{K}(\ell^{2}(P,\mathcal{K}))$, hence has $K_{0}(C^{*}P) \cong \prod \frac{\mathbb{Z}}{K}$, where $K$ is the subgroup of $\bigoplus \mathbb{Z}$ of elements with sum equal to 0. Now:

To prove (1), Observe that the following diagram commutes:
\begin{equation*}
\xymatrix{ K_{0}(C^{*}X) \ar[r]& \frac{\mathbb{Z}}{K}\\
K_{0}(\mathcal{K}) \ar[ur]\ar[ul]
}
\end{equation*}
Where the image of a rank one projection in $K_{0}(\mathcal{K})$ is given by a vector $(1,0,...)$, which is certainly injection. It now follows that the upward map is injective.

For (2) consider a nontrivial ghost projection $p \in I_{G}$. It is well known that $Tr([p]) \not\in \bigoplus \mathbb{Z}$, whilst $Tr([k]) \in \bigoplus(\mathbb{Z})$ for any compact operator \cite{explg1}. As these differ under $Tr$, they cannot possibly be equal in $K_{0}(I_{G})$.
\end{proof}

########################################

We present a new approach to study the coarse Baum-Connes conjecture for spaces of graphs of bounded geometry, specifically providing new insight into counterexamples from \cite{higsonpreprint,MR1911663,explg1}. Recall that the coarse Baum-Connes conjecture states that a certain assembly map:
\begin{equation*}
\mu_{X,red}:KX_{*}(X) \longrightarrow K_{*}(C^{*}(X))
\end{equation*}
is an isomorphism for $X$ a uniformly discrete bounded geometry metric space. This conjecture is a geometric intrepretation of the well-known \textit{Baum-Connes conjecture} \cite{MR1292018}. For finitely generated groups the Cayley graph coming from some finite generating set will be a uniformly discrete space with bounded geometry, and a positive result for the coarse Baum-Connes conjecture in such situations has strong implications such as the Strong Novikov conjecture \cite{MR866507}, or the existence of metrics with positive scalar curvature.

The Baum-Connes conjecture can be developed in other directions, particularly into the realm of topological groupoids \cite{MR1798599}, which arise naturally in topology and noncommutative geometry to study objects such as foliations on manifolds or group actions on topological spaces. It is a well known result from \cite{MR1905840} that the above statement of the coarse Baum-Connes conjecture can be replaced with a conjecture with coefficients for some groupoid $G(X)$ that we can associate to any uniformly discrete bounded geometry metric space.

In this context, the coarse Baum-Connes conjecture states that the map:
\begin{equation*}
\mu_{r}:K_{*}^{top}(G(X), \ell^{\infty}(X,\mathcal{K})) \rightarrow K_{*}(\ell^{\infty}(X,\mathcal{K})\rtimes_{r}G(X))
\end{equation*}
is an isomorphism. In the paper \cite{MR1911663}, counterexamples were constructed for the coarse groupoid when the space was a \textit{expander graph} \cite{MR2569682}. This class of spaces has subsequently been well studied with respect to the coarse Baum-Connes conjecture \cite{higsonpreprint,MR2431253,MR2419930,MR2764895,MR2568691,explg1,explg2}. The key property associated to expander graphs is the ability to construct \textit{ghost projections} in the Roe algebra. These are infinite rank projections that cannot be approximated well by operators of very small propagation as they have essentially no local information. 

In the paper \cite{MR1911663} the method for constructing counterexamples to the conjecture for groupoids utilised a ``short exact sequence'' built from reductions of a groupoid -the main idea was that this sequence, whilst always exact for the maximal $C^{*}$-algebras of these groupoids may fail to be exact when we consider the left regular representation. The spotlight in \cite{MR1911663} was on the original groupoid, not the reductions, whilst in this paper we are going to consider specifically the groupoids that arise from reductions to \textit{boundaries} of the unit space. 

The object of this paper to give elementary proofs of many of these results concerning \textit{box spaces} of residually finite groups as well as proving a strengthened geometric version of the main results of Willett and Yu in \cite{explg1,explg2}. We obtain this framework by revisiting the ideas of \cite{MR1911663,MR1905840} and by introducing a boundary conjecture associated to a uniformly discrete bounded geometry metric space $X$:
\begin{conjecture}(Boundary Coarse Baum-Connes conjecture)\label{MC:S1}
Let $X$ be a uniformly discrete bounded geometry metric space and let $G(X)$ be the associated coarse groupoid on $X$. Then:
\begin{equation*}
\mu_{X,bdry}:K_{*}^{top}(G(X)|_{\partial\beta X}, l^{\infty}(X,\mathcal{K})/C_{0}(X,\mathcal{K})) \rightarrow K_{*}((l^{\infty}(X,\mathcal{K})/C_{0}(X,\mathcal{K}))\rtimes_{r}G(X)|_{\partial\beta X})
\end{equation*}
is an isomorphism.
\end{conjecture}

The intuition behind this conjecture is to quotient out by the ghost operators in the Roe algebra and then consider the K-theory of what remains. We formalise this idea in section \ref{Sect:CE}.

Now consider a space $X = \sqcup X_{i}$ constructed from a sequence of finite graphs $\lbrace X_{i}\rbrace$. The boundary coarse Baum-Connes conjecture for $X$ in this instance allows us, via the Five Lemma and \cite[Lemma 9]{MR1905840}, to conclude the coarse Novikov conjecture for $X$. We illustrate how this setting can be adapted to provide new proofs of results in \cite{explg1} concerning large girth expanders as well as the result of \cite[Theorem 5.7]{MR2764895} concerning box spaces of linear groups. The maximal conjecture is also considered thereby allowing us to get a new proof a Theorem of \cite{MR2431253,MR2419930,MR2568691}. This relationship makes it desirable to understand for which sequences this boundary conjecture is an isomorphism and to that end we prove:

\begin{theorem}\label{MT:S1}
Conjecture \ref{MC:S1} is true for the following classes of spaces:
\begin{enumerate}
\item Sequences of spaces that uniformly uniformly embed into Hilbert Space. 
\item Sequences of spaces that have large girth and uniformly bounded vertex degree.
\item Generalised Box Spaces associated to finitely generated groups with the Strong Baum-Connes Property.
\end{enumerate}
\end{theorem}
This theorem allows us to recover all known information about spaces of the form $X = \sqcup X_{i}$ for sequences of finite graphs $\lbrace X_{i} \rbrace$. Remarkably this includes some examples of groups with property (T), whose box spaces have geometric property (T) introduced by Willett and Yu in \cite{explg2}.

In the last section of this paper we provide a counterexample to the boundary coarse Baum-Connes conjecture defined in conjecture \ref{MC:S1}. We now recap all the major themes.